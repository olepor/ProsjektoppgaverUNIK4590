
\documentclass[10pt,a4paper,sans]{article} 

\usepackage[utf8]{inputenc}

\title{Prosjektoppgave2 UNIK4590 - h\o st 2017}
\author{Ole Petter Orhagen}
\date{\today}

\begin{document}
\maketitle
\newpage

\section*{Bruksanvisning}
Hele oppgaven er i utgangspunktet l\o st i fila \emph{oppg2.m} i kode mappa.
kj\o r denne fila i matlab, og velg ut de tre regionene du vil klassifisere
etter, og ta deg en kopp kaffe (klassifiseringene tar litt tid). S\aa\ vil kj\o re klassifikatoren p\aa\ treningsbildet \emph{bilde2}, og
p\aa\ \emph{bilde3} for \aa\ klassifisere noe som ikke er et treningsbilde.
(P.S: bildet der man velger ut de tre regionene man vil klassifisere etter
lukker seg ikke selv, og m\aa\ derfor lukkes manuelt for at programmet skal
fortsette \aa\ kj\o re. Det er ogs\aa\ per n\aa\ kun mulighet til \aa\ velge tre
regioner \aa\ klassifisere etter.)

\section*{Oppsummering}
Klassifikatoren ser ut til \aa\ fungere ganske godt, bortsett fra at m\o rke
regioner fort blir klassifisert til bl\aa tt, og lysere udefinerte regioner blir
klassifisert til gulvet. Dette er fordi vi kun har tra klassifikatorer \aa\
legge et piksel i. Vi kunne for eksempel satt en \emph{sikkerhetsmargin} ved \aa\ for
eksempel ha en \emph{troverdighet} p\aa\ hver klassifisering, og dermed, dersom
ingen av klassifikatorene var sikre nok, s\aa\ vil vi klassifisere den som uklassifisert.

\end{document}